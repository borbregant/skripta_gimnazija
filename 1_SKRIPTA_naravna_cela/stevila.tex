\documentclass{article}
\usepackage[utf8]{inputenc}
\usepackage{pgfplots}
\pgfplotsset{width=10cm,compat=1.9}
\usepackage{amsmath,amssymb,amsthm}
\usepackage{graphicx}
\usepackage{float}
\usepackage{blindtext}
\usepackage{hyperref}
\usepackage{verbatim}
\hypersetup{
    colorlinks=true,
    linkcolor=blue,
    filecolor=magenta,      
    urlcolor=cyan,
    pdftitle={Overleaf Example},
    pdfpagemode=FullScreen,
    }
\usepackage[slovene]{babel}

\newcounter{example}[section]
\newenvironment{example}[1][]{\refstepcounter{example}\par\medskip
   \noindent \textbf{Naloga~\theexample. #1} \rmfamily}{\medskip}

\newtheorem*{zgled}{Zgled}

\title{Naravna in cela števila}
\author{Bor Bregant}
\date{\vspace{-5ex}}

\begin{document}

\maketitle

\section{Naravna števila}

Števila s katerimi štejemo $\mathbb{N}=\{1,2,3,\ldots\}$.\\
Peanovi aksiomi:
\begin{itemize}
    \item $1\in\mathbb{N}$
    \item Vsako naravno število ima svojega naslednika
    \item Različni naravni števili imata različna naslednika
    \item Če neka trditev velja z vsakim naravnim številom tudi za njegovega naslednika, velja za vsa naravna števila.
\end{itemize}

Osnovni operaciji $+$ in $\cdot$ (notranji). Seštevanec, vsota, faktor, produkt.\\
Komutativnost $a+b=b+a \ ,\cdot$, asociativnost $(a+b)+c=a+(b+c)$, distributivnost $(a+b)c=ac+bc$.


\begin{zgled}
    Izračunaj $75\cdot 3-12+16\cdot (-5)$ in $2+7\cdot 3(2+4(3+2\cdot 2(5-7\cdot 8)))$ ter $172\cdot 29$.
\end{zgled}

\begin{zgled}
    Odpravi oklepaje $7(3x+1)$ in $(3a+4)(5b+2)$
\end{zgled}

\begin{zgled}
    Izpostavi skupni faktor $10a+30$ in $ac+bc+a+b$.
\end{zgled}


\begin{example}
    5a, 8ce, 9be
\end{example}

\section{Cela števila}

Dodamo nasprotna števila $n\rightarrow -n$. $\mathbb{Z}$ konstruiramo kot unija pozitivnih celih števil, števila $0$ in negativnih celih števil.

Dodamo $-$, ki je definirano kot kot prištevanje nasprotne vrednosti.

Nekaj aksiomov in pravil, urejenost. Vrstni red pri računanju.

Aksiom: $a+0=a$, $-(-a)=a$, $1\cdot a =a$ $\forall a\in\mathbb{Z}$\\
Izrek: $-(a)+(-b)=-(a+b)$, $0\cdot a =0$.

Urejenost (primerjamo števila): Velja natanko ena od možnosti $a<b,a>b,a=b$. $a>b$ če in samo če $a-b>0$ (slika $a$ leži na desni strani številske premice od števila $b$)

\begin{zgled}
    Trikratniku števila $62$ odštejemo petkratnik vsote števil $93$, $82$ in $8$. Katero število dobimo?
\end{zgled}

\begin{zgled}
    Zapiši množico vseh celih števil, ki so od $0$ oddaljena kvečjemu za $6$, ter iskano množico predstavi na številski premici.
\end{zgled}

\section{Potence z naravnimi eksponenti}

\[a^n=a\cdot\ldots\cdot a\]

Osnova, eksoinent, potenca

Pravila z dokazi:
\[a^n\cdot a^m =a^{n+m}\]
\[\left(a^n\right)^m=a^{n\cdot m}\]
\[\left(ab\right)^n=a^n b^n\]
\[a^1=a, \ 1^n = 1\]

\begin{zgled}
    Izračunaj $x^2 \cdot x^9 +2x\cdot x^{10}$, $\left(a^n\right)^2\cdot\left(a^3 \right)^n$, $\left(u^2v^3\right)^2$, $\left(a^2b\right)^3\left(3ab^3\right)^2a^2$ in $(-1)^{2023}\cdot (-1)^{2024}$.
\end{zgled}

\begin{example}
    82cg, 72ab, 80d, 90a
\end{example}

\section{Večkratniki in izrazi}

Večkratnik ali $k$-kratnik števila $a$ je vsota $k$ enakih sumandov $a$: $k\cdot a = a+\ldots +a$.

\[(a+b)^2=a^2+2ab+b^2 \ \text{kvadrat vsote}\]
\[(a-b)^2=a^2-2ab+b^2\]
\[(a+b)^3=a^3+3a^2b+3ab^2+b^3\]
\[(a-b)^3=a^3-3a^2b+3ab^2-b^3\]
\[a^2-b^2=(a+b)(a-b)\]
\[a^3\pm b^3=(a\pm b)(a^2\mp ab+b^2)\]
\[ab+ac =a(b+c)\]

\begin{zgled}
    Izračunaj $(2a+3b)^2$,...
\end{zgled}

\[a^n-b^n=(a-b)(a^{n-1}+a^{n-2}b+a^{n-3}b^2+\ldots + ab^{n-2}+b^{n-1})\]

\begin{zgled}
    Izračunaj $a^7-1$
\end{zgled}

\begin{example}
    Linea nova
\end{example}

Vietovo pravilo

\end{document}