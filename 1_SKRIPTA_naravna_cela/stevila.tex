\documentclass{article}
\usepackage[utf8]{inputenc}
\usepackage{pgfplots}
\pgfplotsset{width=10cm,compat=1.9}
\usepackage{amsmath,amssymb,amsthm}
\usepackage{graphicx}
\usepackage{float}
\usepackage{blindtext}
\usepackage{hyperref}
\usepackage{verbatim}
\usepackage{graphicx}
\hypersetup{
    colorlinks=true,
    linkcolor=blue,
    filecolor=magenta,      
    urlcolor=cyan,
    pdftitle={Overleaf Example},
    pdfpagemode=FullScreen,
    }
\usepackage[slovene]{babel}

\setlength{\parindent}{0pt}
\setlength{\parskip}{4pt}

\newcounter{example}[section]
\newenvironment{example}[1][]{\refstepcounter{example}\par\medskip
   \noindent \textbf{Naloga~\theexample. #1} \rmfamily}{\medskip}

\newtheorem*{zgled}{Zgled}

\title{Naravna in cela števila}
\author{Bor Bregant}
\date{\vspace{-5ex}}

\begin{document}

\maketitle

\section{Naravna števila}

Števila s katerimi štejemo $\mathbb{N}=\{1,2,3,\ldots\}$.

Peanovi aksiomi:
\begin{itemize}
    \item $1\in\mathbb{N}$
    \item Vsako naravno število ima svojega naslednika
    \item Različni naravni števili imata različna naslednika
    \item Če neka trditev velja z vsakim naravnim številom tudi za njegovega naslednika, velja za vsa naravna števila.
\end{itemize}

Osnovni operaciji $+$ in $\cdot$ (notranji). Seštevanec, vsota, faktor, produkt.

Komutativnost $a+b=b+a \ ,\cdot$\\
Asociativnost $(a+b)+c=a+(b+c)$\\
Distributivnost $(a+b)c=ac+bc$.


\begin{zgled}
    Izračunaj $75\cdot 3-12+16\cdot (-5)$ in $2+7\cdot 3(2+4(3+2\cdot 2(5-7\cdot 8)))$ ter $172\cdot 29$.
\end{zgled}

\begin{zgled}
    Odpravi oklepaje $7(3x+1)$ in $(3a+4)(5b+2)$
\end{zgled}

\begin{zgled}
    Izpostavi skupni faktor $10a+30$ in $ac+bc+a+b$.
\end{zgled}


\begin{example}
    5a, 8ce, 9be
\end{example}

\section{Cela števila}

Dodamo nasprotna števila $n\rightarrow -n$. $\mathbb{Z}$ konstruiramo kot unija pozitivnih celih števil, števila $0$ in negativnih celih števil.

Dodamo $-$, ki je definirano kot kot prištevanje nasprotne vrednosti.

Nekaj aksiomov in pravil:

Aksiom: $a+0=a$, $-(-a)=a$, $1\cdot a =a$ $\forall a\in\mathbb{Z}$

Izrek: $-(a)+(-b)=-(a+b)$, $0\cdot a =0$.

Urejenost (primerjamo števila): Velja natanko ena od možnosti $a<b,a>b,a=b$. $a>b$ če in samo če $a-b>0$ (slika $a$ leži na desni strani številske premice od števila $b$).

Za relacijo "biti manjši ali enak" (in podobno za $\geq$) veljajo lastnosti:

Refleksivnost $a\leq a$,\\
Antisimetričnost $a\leq b \land b\leq a \Rightarrow a=b$,\\
Tranzitivnost $a\leq b \land b\leq c \Rightarrow a\leq c$.


\begin{zgled}
    Trikratniku števila $62$ odštejemo petkratnik vsote števil $93$, $82$ in $8$. Katero število dobimo?
\end{zgled}

\begin{zgled}
    Zapiši množico vseh celih števil, ki so od $0$ oddaljena kvečjemu za $6$, ter iskano množico predstavi na številski premici.
\end{zgled}

\section{Potence z naravnimi eksponenti}

\[a^n=\underbrace{a\cdot\ldots\cdot a}_{n}\]

Osnova, eksponent, potenca

Pravila z dokazi:
\begin{align*} 
a^n\cdot a^m &=a^{n+m}\\
\left(a^n\right)^m&=a^{n\cdot m}\\
\left(ab\right)^n&=a^n b^n\\
a^1&=a, \ 1^n = 1
\end{align*}

\begin{zgled}
    Izračunaj $x^2 \cdot x^9 +2x\cdot x^{10}$, $\left(a^n\right)^2\cdot\left(a^3 \right)^n$, $\left(u^2v^3\right)^2$, $\left(a^2b\right)^3\left(3ab^3\right)^2a^2$ in $(-1)^{2023}\cdot (-1)^{2024}$, $(2x^3)^2\cdot (-3x^4)^2$.
\end{zgled}

\begin{example}
    82cg, 72ab, 80d, 90a
\end{example}

\section{Večkratniki in izrazi}

Večkratnik ali $k$-kratnik števila $a$ je vsota $k$ enakih sumandov $a$: $k\cdot a = a+\ldots +a$.

\begin{align*}
(a+b)^2&=a^2+2ab+b^2 \ \text{kvadrat vsote}\\
(a-b)^2&=a^2-2ab+b^2 \ \text{kvadrat razlike}\\
(a+b)^3&=a^3+3a^2b+3ab^2+b^3 \ \text{kub vsote}\\
(a-b)^3&=a^3-3a^2b+3ab^2-b^3 \ \text{kub razlike}\\
a^2-b^2&=(a+b)(a-b) \ \text{razlika kvadratov}\\
a^3\pm b^3&=(a\pm b)(a^2\mp ab+b^2) \ \text{vsota (razlika) kubov}\\
ab+ac &=a(b+c) \ \text{izpostavljanje skupnega faktorja}
\end{align*} 

\begin{zgled}
    Izračunaj $(2a+3b)^2$, $(x-2y)^2$, $(3u+v)^3$, $(2-5n)^3$ in $(x^2-2x-3)(x^2+3x)$.
\end{zgled}

\begin{zgled}
    Zapiši prve tri večkratnike izraza $x-2$.
\end{zgled}

\begin{zgled}
    Razstavi $x^2+2x+1$, $a^2-9$, $16a^2-81$, $25+10a+a^2$, $x^3+64y^3$, $3a+6a^2$, $ac+ad+bc+bd$ in $2x^2-2xz+xy-yz$.
\end{zgled}

\begin{zgled}
    Razstavi $x^4-3x^3+2x^2$, in $a^4b+64ab$.
\end{zgled}

\begin{example}
    DN 114a, 122a, 123a, 124acd, 125acd, 130a, 131ch, 135a
\end{example}

\begin{align*}
a^n-b^n&=(a-b)(a^{n-1}+a^{n-2}b+a^{n-3}b^2+\ldots + ab^{n-2}+b^{n-1})\\
a^{2n+1}+b^{2n+1}&=(a+b)(a^{2n}-a^{2n-1}b+a^{2n-2}b^2+\cdots -ab^{2n-1}+b^{2n}) \ \text{lihi naravni eksponenti}
\end{align*}

\begin{zgled}
    Izračunaj $a^7-1$, $a^5+32b^5$.
\end{zgled}

Vietovo pravilo
\[x^2+(a+b)x+ab=(x+a)(x+b)\]

\begin{zgled}
    Razstavi $x^2+5x+6$, $x^2-11x+18$, $m^2+7m-8$ in $a^4+a^2-20$.
\end{zgled}

\begin{zgled}
    Razstavi $a^2+10ab+24b^2$ in $x^3+x^2+3x+3$.
\end{zgled}

\begin{example}
    DN 127ac, 128b, 132a
\end{example}

\end{document}