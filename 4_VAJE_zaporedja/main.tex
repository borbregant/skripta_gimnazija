\documentclass{article}
\usepackage[utf8]{inputenc}
\usepackage{pgfplots}
\pgfplotsset{width=10cm,compat=1.9}
\usepackage{amsmath,amssymb,amsthm}
\usepackage{gensymb}
\usepackage{graphicx}
\usepackage{amsmath}
\usepackage{float}
\usepackage{xcolor}
\usepackage{blindtext}
\usepackage[margin=1in]{geometry}
\usepackage{hyperref}
\hypersetup{
    colorlinks=true,
    linkcolor=blue,
    filecolor=magenta,      
    urlcolor=cyan,
    pdftitle={Overleaf Example},
    pdfpagemode=FullScreen,
    }
\usepackage[slovene]{babel}

\setlength{\parindent}{0pt}
\setlength{\parskip}{4pt}


\newcounter{example}[section]
\newenvironment{example}[1][]{\refstepcounter{example}\par\medskip
   \noindent \textbf{Naloga~\theexample. #1} \rmfamily}{\medskip}


\title{Zaporedja vaje}
\author{Bor Bregant}
\date{\vspace{-5ex}}

\begin{document}

\maketitle

\section{Osnovni nivo}

\begin{example}
    V geometrijskem zaporedju je tretji člen enak $40$, šesti pa $320$.

    Ali je število $81900$ člen danega zaporedja? Odgovor utemelji.

    Koliko začetnih členov tega zaporedja moramo sešteti, da dobimo vsoto 20470?
\end{example}

\begin{example}
    Dano je aritmetično zaporedje s splošnim členom $a_n=2n-2$.

    Izračunaj vsoto $\sum_{n=1}^{100}a_n$.

    Dokaži, da je zaporedje, dano s splošnim členom $b_n=2^{a_n}$ geometrijsko.
\end{example}

\begin{example}
    $27$, $9$ in $3$ so prvi trije členi geometrijskega zaporedja. Zapiši četrti člen in količnik $q$.
\end{example}

\section{Višji nivo}

\begin{example}
    Notranji koti trikotnika so zaporedni členi aritm. zap. Dokaži $a^2-ac+c^2=b^2$.
\end{example}

\begin{example}
    Trije zaporedni členi narašč. geom. zap. imajo vsoto $52$. Če prvemu prištejemo $1$, drugemu $8$, tretjega pa zmanjšamo za $1$ dobimo zaporedne člene aritm. zap. Izračunaj prve tri člene.
\end{example}

\begin{example}
    Pokaži $\sum_{k=1}^{n}\frac{1}{k(k+1)}=\frac{n}{n+1}$.
\end{example}

\begin{example}
    Izračunaj $\lim_{n\rightarrow\infty}\left(\sqrt{n+1}-\sqrt{n-1}\right)$ in $\lim_{n\rightarrow\infty}\left(\sqrt{n^2-2n}-\sqrt{n^2+n}\right)$.
\end{example}

\end{document}