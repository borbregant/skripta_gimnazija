\documentclass{article}
\usepackage[utf8]{inputenc}
\usepackage{pgfplots}
\pgfplotsset{width=10cm,compat=1.9}
\usepackage{amsmath,amssymb,amsthm}
\usepackage{graphicx}
\usepackage{float}
\usepackage{blindtext}
\usepackage{hyperref}
\usepackage{verbatim}
\usepackage{gensymb}
\usepackage{enumerate}
\usepackage{xcolor}
\usepackage{graphicx}
\hypersetup{
    colorlinks=true,
    linkcolor=blue,
    filecolor=magenta,      
    urlcolor=cyan,
    pdftitle={Overleaf Example},
    pdfpagemode=FullScreen,
    }
\usepackage[slovene]{babel}

\setlength{\parindent}{0pt}
\setlength{\parskip}{4pt}

\newcommand{\Lim}[1]{\raisebox{0.5ex}{\scalebox{0.8}{$\displaystyle \lim_{#1}\;$}}}
\newcounter{example}[section]
\newenvironment{example}[1][]{\refstepcounter{example}\par\medskip
   \noindent \textbf{Naloga~\theexample. #1} \rmfamily}{\medskip}

\newtheorem*{zgled}{Zgled}

\title{Racionalna števila}
\author{Bor Bregant}
\date{\vspace{-5ex}}

\begin{document}

\maketitle

\section*{Ulomki in decimalni zapis}

Desetiški ulomek: V decimalnem zapisu je število decimalk končno

\begin{zgled}
    $27:40=\textcolor{blue}{0}.\textcolor{green}{675}=\frac{675}{1000}$ in $67:16$.
\end{zgled}

Končen decimalni zapis $\iff$ imenovalec ulomka potenca $10$ $\iff$ \textcolor{red}{v $\mathbb{P}$ razcepu imenovalca le 2 in 5.}

Primeri: $\frac{3}{40}$, $\frac{1}{2}$ in $\frac{3}{50}$.

Nedesetiški ulomek: Deljenje se ne konča v končno korakih.

Primeri: $\frac{2}{3}=0.666\ldots=0.\bar{6}$, $14:11=1.2727\ldots$. Ponavljanje imenujemo perioda reda $r$.

\begin{zgled}
    Zapiši $\frac{13}{100}$ in $\frac{15}{8}$ v decimalnem ter $1.6$ in $-0.04$ v ulomku.
\end{zgled}

\begin{zgled}
    Na pamet izračunaj $1.27\cdot 10$, $4.5:100$ in $3.12\cdot 10^{-1}$.
\end{zgled}

\begin{zgled}
    Zapiši $3.\bar{6}$, $2.\bar{45}$ in $1.4\bar{72}$ \textcolor{gray}{najprej $\cdot 10$, potem $\cdot 1000$} v obliki ulomka.
\end{zgled}

\begin{zgled}
    Zapiši $\frac{2}{3}$, $\frac{3}{11}$ v decimalnem in $0.\bar{202}$ v obliki ulomka.
\end{zgled}

\begin{zgled}
    Izračunaj $0.125+0.25\cdot \frac{8}{5}$.
\end{zgled}

\begin{example}
    Ponovi teorijo, 370 bd, 371abfg, 373b
\end{example}

\end{document}