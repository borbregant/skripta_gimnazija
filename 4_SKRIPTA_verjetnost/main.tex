\documentclass{article}
\usepackage[utf8]{inputenc}
\usepackage{pgfplots}
\pgfplotsset{width=10cm,compat=1.9}
\usepackage{amsmath,amssymb,amsthm}
\usepackage{gensymb}
\usepackage{graphicx}
\usepackage{float}
\usepackage{xcolor}
\usepackage{blindtext}
\usepackage{hyperref}
\usepackage{enumerate}
\usepackage[margin=1in]{geometry}
\hypersetup{
    colorlinks=true,
    linkcolor=blue,
    filecolor=magenta,      
    urlcolor=cyan,
    pdftitle={Overleaf Example},
    pdfpagemode=FullScreen,
    }
\usepackage[slovene]{babel}

\newcounter{example}[section]
\newenvironment{example}[1][]{\refstepcounter{example}\par\medskip
   \noindent \textbf{Naloga~\theexample. #1} \rmfamily}{\medskip}

\newtheorem*{zgled}{Zgled}

\title{Verjetnost}
\author{Bor Bregant}
\date{\vspace{-5ex}}

\begin{document}

\maketitle

\textbf{Poskus} $\rightarrow$ \textbf{dogodek} $\rightarrow$ nemogoč, gotov, \fbox{slučajen}

\textbf{Elementarni in sestavljeni dogodek} (npr. pade liho število pik na kocki)

Množice in dogodki:

Unija - Vsota dogodkov (enaka oznaka u)

\begin{zgled}
    Na koliko načinov lahko za ravno mizo sedi sedem povabljencev?
\end{zgled}





\end{document}