\documentclass{article}
\usepackage[utf8]{inputenc}
\usepackage{pgfplots}
\pgfplotsset{width=10cm,compat=1.9}
\usepackage{amsmath,amssymb,amsthm}
\usepackage{gensymb}
\usepackage{graphicx}
\usepackage{float}
\usepackage{xcolor}
\usepackage{blindtext}
\usepackage{hyperref}
\usepackage{enumerate}
\usepackage[margin=1in]{geometry}
\hypersetup{
    colorlinks=true,
    linkcolor=blue,
    filecolor=magenta,      
    urlcolor=cyan,
    pdftitle={Overleaf Example},
    pdfpagemode=FullScreen,
    }
\usepackage[slovene]{babel}
\setlength{\parindent}{0pt}
\setlength{\parskip}{4pt}

\newcounter{example}[section]
\newenvironment{example}[1][]{\refstepcounter{example}\par\medskip
   \noindent \textbf{Naloga~\theexample. #1} \rmfamily}{\medskip}

\newtheorem*{zgled}{Zgled}

\title{Verjetnost}
\author{Bor Bregant}
\date{\vspace{-5ex}}

\begin{document}

\maketitle

\textbf{Poskus} $\rightarrow$ \textbf{dogodek} $\rightarrow$ nemogoč, gotov, \fbox{slučajen}

\textbf{Elementarni in sestavljeni dogodek} (npr. pade liho število pik na kocki)

Množice in dogodki (lastnosti komutativnost):

Unija - Vsota dogodkov (enaka oznaka $\cup$)

Presek ($\{x;x\in A \land x\in B\}$) - Produkt dogodkov (enaka oznaka $\cap$). Primer $A$: manj kot 3 pike, $B$: liho število pik $\rightarrow$ $A\cap B$: pade 1. Komut., $A\cap G=A$, $A\cap N=N$.

Množici disjunktni $\rightarrow$ - Nezdružljiva dogodka npr. $A$ dve piki, $B$ pet pik $\rightarrow$ $A\cap B=N$.

Komplementarna množica $A^c$ - Nasprotni dogodek $A'$ in se zgodi, ko se $A$ ne zgodi. $A\cap A'=N$, $G' =N$.

Razlika dogodkov $A-B$: $A$ zgodi, $B$ se ne zgodi, ni komutativna

Podmnožica - Način dogodka: $A\subset B$: Vsakič, ko se zgodi $A$, se zgodi tudi $B$.

\begin{zgled}
    $A$ naj bo izvlečem rdečo karto, $B$ naj bo izvlečem srčevega kralja. Kakšna zveza velja?
\end{zgled}

Met kocke ima $6$ elementarnih dogodkov. Iz tega sestavimo \textbf{vzorčni prostor}. Sestavljajo ga dogodki, ki so med seboj nezdružljivi in je njihova vsota gotov dogodek.
Še en vzorčni prostor bi lahko bil $A$ pade sodo pik, $B$ pade liho pik.

Vzorčni prostor dveh kock lahko predstavimo kot mrežo vseh možnosti. Pomaga pri npr. koliko verjetnost, da pade $5$ pik.

Vzorčni prostor lahko tudi z drevesom npr. iz vrečke jemljemo kroglice bele in črne. Dobimo 4 končne veje $BB$, $BC$, $CB$, $CC$.

\textbf{Empirična definicija verjetnost}: Verjetnost dogodka $A$ je enaka relativni frekvenci dogodka $A$ pri dovolj velikem številu ponovitev poskusa. $f_A =\frac{n_A}{n}$.

\textbf{Klasična definicija verjetnosti}: Če so vsi elementarne dogodki nekega poskusa enakovredni in je $A$ dogodek iz vzorčnega prostora tega poskusa, je verjetnost enaka $P(A)=\frac{m}{n}=\frac{\text{st. elementarni dogodkov, ki so ugodni za $A$}}{\text{st. vseh elementarnih dogodkov}}$.

\begin{zgled}
    Kocka iz $E_1 \ldots E_6$ in naj je $A$: padejo tri pike. $P(A)=\frac{1}{6}$. $B$ pade sodo pik.
\end{zgled}

\begin{enumerate}[i]
    \item $P(A)\geq 0$
    \item $P(G)=1$
    \item $P(A\cup B)=P(A)+P(B)$, če sta $A$ in $B$ nezdružljiva, torej $A\cap B=N$.
\end{enumerate}

\begin{zgled}
    V posodi je $20$ oštevilčenih listkov od $1$ do $20$. Izvlečemo en listek. Kolikšna je verjetnost za:\\
    $A$: izvlečeno število je sodo\\
    $B$: Izvlečeno število je deljivo s $3$ ali s $5$.\\
    $C$: Izvlečeno število ni večkratnik $3$.
\end{zgled}

\begin{zgled}
    Imamo $3$ kovance za $10$, $20$ in $50$ centov. Vržemo jih v zrak in pade cifra ali mož. Nariši vzorčni prostor.\\
    $A$: Nobeden ne pokaže cifre\\
    $B$: Cifro pokaže eden od treh kovancev\\
    $C$: Cifro pokažeta dva od treh\\
    $D$: Cifro pokažeta vsaj dva\\
\end{zgled}

\begin{zgled}
    Imamo pošteno kocko, ki pa ima stranice: 1, 1, 2, 1, 3, 4. Kolikšna je verjetnost, da pade ena pika, da pade 6 pik, da padejo 3 pike
\end{zgled}

\begin{zgled}
    Na dirki tekmujejo konji $A$, $B$. $C$. $A$ ima pol možnosti glede na $B$, konj $B$ pa ima trikrat večjo možnost zmage kot $C$. Koliko so verjetnosti za zmago posameznega konja.
\end{zgled}

....



\section*{Pogojna verjetnost}

Dogodka $A$ in $B$ sta \textbf{neodvisna}, če verjetnost enega ne vpliva na verjetnost drugega. Ekvivalentno $P\left(A\cap B\right) =P(A)\cdot P(B)$.

Dogodek $B$ je odvisen od dogodka $A$, če je verjetnost $B$ odvisna od tega, ali se je $A$ zgodil ali ne. Pišemo $P(B|A)$.

Če sta dogodka neodvisna, je $P(B|A)=P(B)$.

\begin{zgled}
    V vreči imamo 7 belih in 3 rdeče kroglice. Kolikšna je verjetnost, da izvlečemo 3 kroglice iste barve, če kroglice vračamo ali pa ne. Če ne vračamo izračunajmo, če vlečemo kroglice zapored ali naenkrat \textcolor{gray}{(verjetnost bo tu enaka)}.
\end{zgled}

$P(B|A)=\frac{P(A\cap B)}{P(A)}$ in slika Vennovega diagrama!!!

\begin{zgled}
    Izmed listkov s števili od 1 do 11 izberemo naključno dve števili. Kolikšna je verjetnost, da sta izbrani števili lihi, če je vsota sodo število.
\end{zgled}

\begin{zgled}
    Mečemo dve kocki. Kolikšna je verjetnost, da je na eni kocki padla 6, če kocki pokažeta vsoto 8.
\end{zgled}

\begin{zgled}
    Izračunaj $P(A\cap B)$, $P(A|B)$ in $P(A' \cap B)$, če je $P(A)=\frac{1}{4}$, $P(B)=\frac{1}{3}$ in $P(A \cup B)=\frac{3}{5}$. \textcolor{gray}{$P(A' \cap B)=P(B)-P(A\cap B)$ iz diagrama}
\end{zgled}

\begin{example}
    DN 476, 483, 512, 521.
\end{example}





\end{document}