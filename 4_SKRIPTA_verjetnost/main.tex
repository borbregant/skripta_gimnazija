\documentclass{article}
\usepackage[utf8]{inputenc}
\usepackage{pgfplots}
\pgfplotsset{width=10cm,compat=1.9}
\usepackage{amsmath,amssymb,amsthm}
\usepackage{gensymb}
\usepackage{graphicx}
\usepackage{float}
\usepackage{xcolor}
\usepackage{blindtext}
\usepackage{hyperref}
\usepackage{enumerate}
\usepackage[margin=1in]{geometry}
\hypersetup{
    colorlinks=true,
    linkcolor=blue,
    filecolor=magenta,      
    urlcolor=cyan,
    pdftitle={Overleaf Example},
    pdfpagemode=FullScreen,
    }
\usepackage[slovene]{babel}
\setlength{\parindent}{0pt}
\setlength{\parskip}{4pt}

\newcounter{example}[section]
\newenvironment{example}[1][]{\refstepcounter{example}\par\medskip
   \noindent \textbf{Naloga~\theexample. #1} \rmfamily}{\medskip}

\newtheorem*{zgled}{Zgled}

\title{Verjetnost}
\author{Bor Bregant}
\date{\vspace{-5ex}}

\begin{document}

\maketitle

\textbf{Poskus} $\rightarrow$ \textbf{dogodek} $\rightarrow$ nemogoč, gotov, \fbox{slučajen}

\textbf{Elementarni in sestavljeni dogodek} (npr. pade liho število pik na kocki)

Množice in dogodki (lastnosti komutativnost):

Unija - Vsota dogodkov (enaka oznaka $\cup$)

Presek ($\{x;x\in A \land x\in B\}$) - Produkt dogodkov (enaka oznaka $\cap$). Primer $A$: manj kot 3 pike, $B$: liho število pik $\rightarrow$ $A\cap B$: pade 1. Komut., $A\cap G=A$, $A\cap N=N$.

Množici disjunktni $\rightarrow$ - Nezdružljiva dogodka npr. $A$ dve piki, $B$ pet pik $\rightarrow$ $A\cap B=N$.

Komplementarna množica $A^c$ - Nasprotni dogodek $A'$ in se zgodi, ko se $A$ ne zgodi. $A\cap A'=N$, $G' =N$.

Razlika dogodkov $A-B$: $A$ zgodi, $B$ se ne zgodi, ni komutativna

Podmnožica - Način dogodka: $A\subset B$: Vsakič, ko se zgodi $A$, se zgodi tudi $B$.

\begin{zgled}
    $A$ naj bo izvlečem rdečo karto, $B$ naj bo izvlečem srčevega kralja. Kakšna zveza velja?
\end{zgled}

Met kocke ima $6$ elementarnih dogodkov. Iz tega sestavimo \textbf{vzorčni prostor}. Sestavljajo ga dogodki, ki so med seboj nezdružljivi in je njihova vsota gotov dogodek.
Še en vzorčni prostor bi lahko bil $A$ pade sodo pik, $B$ pade liho pik.

Vzorčni prostor dveh kock lahko predstavimo kot mrežo vseh možnosti. Pomaga pri npr. koliko verjetnost, da pade $5$ pik.


\end{document}