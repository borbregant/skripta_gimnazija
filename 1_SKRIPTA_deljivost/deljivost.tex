\documentclass{article}
\usepackage[utf8]{inputenc}
\usepackage{pgfplots}
\pgfplotsset{width=10cm,compat=1.9}
\usepackage{amsmath,amssymb,amsthm}
\usepackage{graphicx}
\usepackage{float}
\usepackage{blindtext}
\usepackage{hyperref}
\usepackage{verbatim}
\hypersetup{
    colorlinks=true,
    linkcolor=blue,
    filecolor=magenta,      
    urlcolor=cyan,
    pdftitle={Overleaf Example},
    pdfpagemode=FullScreen,
    }
\usepackage[slovene]{babel}

\newcounter{example}[section]
\newenvironment{example}[1][]{\refstepcounter{example}\par\medskip
   \noindent \textbf{Naloga~\theexample. #1} \rmfamily}{\medskip}

\newtheorem*{zgled}{Zgled}

\title{Deljivost}
\author{Bor Bregant}
\date{\vspace{-5ex}}

\begin{document}

\maketitle

\section{Relacija deljivosti}


\section{Kriteriji deljivosti}

Izpeljava deljivost, da izpostavljamo $10^i$, razcepino na prafaktorje $10^i$. S to izpeljavo pokrijemo 2, 4, 5, 8

Deljivost s $3$:

\begin{align*}
\overline{a_4 a_3 a_2 a_1 a_0} &=\\
 &=a_4 \cdot 10000 + \ldots + a_0\\
 &=a_4 (9999+1)+a_3(999+1)+a_2(99+1)+a_1(9+1)+a_0\\
 &=9999a_4 + a_4 + \ldots + 9a_1 +a_1 +a_0\\
 &=9999a_4+999a_3+99a_2+9a_1 +a_4+a_3+a_2+a_1+a_0\\
 &=9(1111a_4 +111a_3+11a_2+a_1)+a_4+a_3+a_2+a_1+a_0\\
\end{align*}

Če bo vsota števk deljiva s $3$ (ali z $9$), bo celotno število deljivo s $3$ (ali z $9$).

\begin{zgled}
    Ali je $32154032$ deljivo s $2,3,4,5,6$.
\end{zgled}

\begin{zgled}
    Določi števko $a$, da bo število $35167a2$ deljivo s $6$. \textcolor{gray}{$a\in\{0,3,6,9\}$}
\end{zgled}

\begin{zgled}
    Določi števki $a$ in $b$, da bo število $1573a4b$ deljivo s $6$.
\end{zgled}


\end{document}