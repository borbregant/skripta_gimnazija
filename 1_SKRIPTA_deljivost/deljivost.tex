\documentclass{article}
\usepackage[utf8]{inputenc}
\usepackage{pgfplots}
\pgfplotsset{width=10cm,compat=1.9}
\usepackage{amsmath,amssymb,amsthm}
\usepackage{graphicx}
\usepackage{float}
\usepackage{blindtext}
\usepackage{hyperref}
\usepackage{verbatim}
\hypersetup{
    colorlinks=true,
    linkcolor=blue,
    filecolor=magenta,      
    urlcolor=cyan,
    pdftitle={Overleaf Example},
    pdfpagemode=FullScreen,
    }
\usepackage[slovene]{babel}

\newcounter{example}[section]
\newenvironment{example}[1][]{\refstepcounter{example}\par\medskip
   \noindent \textbf{Naloga~\theexample. #1} \rmfamily}{\medskip}

\newtheorem*{zgled}{Zgled}

\title{Deljivost}
\author{Bor Bregant}
\date{\vspace{-5ex}}

\begin{document}

\maketitle

\section{Relacija deljivosti}

\[a|b \iff b=k\cdot a\]

\section{Kriteriji deljivosti}

Izpeljava deljivost, da izpostavljamo $10^i$, razcepino na prafaktorje $10^i$. S to izpeljavo pokrijemo 2, 4, 5, 8

Deljivost s $3$:

\begin{align*}
\overline{a_4 a_3 a_2 a_1 a_0} &=\\
 &=a_4 \cdot 10000 + \ldots + a_0\\
 &=a_4 (9999+1)+a_3(999+1)+a_2(99+1)+a_1(9+1)+a_0\\
 &=9999a_4 + a_4 + \ldots + 9a_1 +a_1 +a_0\\
 &=9999a_4+999a_3+99a_2+9a_1 +a_4+a_3+a_2+a_1+a_0\\
 &=9(1111a_4 +111a_3+11a_2+a_1)+a_4+a_3+a_2+a_1+a_0\\
\end{align*}

Če bo vsota števk deljiva s $3$ (ali z $9$), bo celotno število deljivo s $3$ (ali z $9$).

\begin{zgled}
    Ali je $32154032$ deljivo s $2,3,4,5,6$.
\end{zgled}

\begin{zgled}
    Določi števko $a$, da bo število $35167a2$ deljivo s $6$. \textcolor{gray}{$a\in\{0,3,6,9\}$}
\end{zgled}

\begin{zgled}
    Določi števki $a$ in $b$, da bo število $1573a4b$ deljivo s $6$.
\end{zgled}

\begin{zgled}
    Določi števki $a$ in $b$, da bo število $504a347b$ deljivo s $36$. \textcolor{gray}{Pazimo, da sta v $36$ razcepu tuja torej $9$, $4$}
\end{zgled}

\begin{zgled}
    Določi števko $a$, da bo število $32a5a4a$ deljivo s $36$.
\end{zgled}

\begin{zgled}
    Določi števko $a$, da bo število $32a5a4a$ deljivo s $3$.
\end{zgled}

\begin{zgled}
    Dokaži $6|n^3-3n^2+2n$ za vsak $n\in\mathbb{N}$. \textcolor{gray}{Razstavimo na 3 zaporedna naravna števila}
\end{zgled}

\begin{zgled}
    Pokaži, da je razlika dveh dvomestnih števil, ki imata zamenjani števki $9|\left(\overline{ab}-\overline{ba}\right)$. \textcolor{gray}{$10a+b-10b-a$}
\end{zgled}

\begin{zgled}
    Poišči dvomestno število, ki je petkratnih vsote svojih števk.
\end{zgled}

\begin{zgled}
    Poišči dvomestno število, ki je dvakratnik produkta svojih števk. \textcolor{gray}{Sprehodimo se po $a=1,\ldots$}
\end{zgled}

\begin{example}
    DN 
\end{example}

\subsection{Praštevila in sestavljena števila}

Praštevila so števila, ki imajo natanko dva delitelja: Število $1$ in samega sebe.\\
Število $1$ ni niti praštevilo, niti sestavljeno število. Število $2$ je edino sodo praštevilo.\\
Praštevil je neskončno mnogo z dokazom \textcolor{gray}{pogledamo $P=p_1 \cdot \ldots \cdot p_n +1$}\\
Eratostenovo sito.

Na koliko načinov lahko $24$ zapišemo kot produkt?

\textbf{Osnovni izrek aritmetike}

...

\section{Osnovni izrek o deljenju}

Če število $a\in\mathbb{N}$ delimo s številom $b\in\mathbb{N}$, potem obstajata dve taki števili $k\in\mathbb{N}$ in $r\in\mathbb{N}_0$, da velja:
\[a=k\cdot b +r; \ 0\leq r<b\]
$a$ imenujemo deljenec, $b$ delitelj, $k$ količnik (kvocient), $r$ ostanek.

Če je $r=0$, potem $b$ deli $a$.

\begin{zgled}
    $52$ deli s $15$ in zapiši osnovni izrek o deljenju
\end{zgled}

\begin{zgled}
    Pri deljenju nekega števila $n$ s številom $13$ dobimo kvocient $7$ in ostanek $8$. Katero število smo delili?
\end{zgled}

\begin{zgled}
    Zapiši vsa naravna števila, ki dajo pri deljenju s $5$ ostanek $1$. \textcolor{gray}{$1,6,11, \ldots$ $\rightarrow$ $n=k\cdot 5 +1$}
\end{zgled}

\begin{zgled}
    Če neko število $n$ delimo z $8$ ostanek 7. Kakšen bo stanek, če delimo $n$ s $4$?
\end{zgled}

\begin{zgled}
    Ostanek števila pri deljenju s $24$ je $19$. Kakšen bo ostanek pri deljenju s $6$?
\end{zgled}

\begin{zgled}
    \textcolor{red}{Če vsoto kvadratov dveh zaporednih števil delimo s $4$ dobimo ostanek $1$. Pokaži, da to velja za vsa števila.}
\end{zgled}

Največji skupni delitelj dveh števil je največje tako število, ki deli obe (ali vsa) števili. Pišemo $D\left(a,b\right)$. (v razcepu vzamemo najmanjše potence)

\begin{zgled}
    Poišči največji skupni delitelj $52$ in $130$ ter $240$ in $186$.
\end{zgled}

\begin{example}
    203, 206, 209, 210, 215
\end{example}

\begin{zgled}
    Poišči največji skupni delitelj $2^5\cdot 3^2\cdot 7^4$ in $2^3\cdot 7^5$ ter $13$ in $16$.
\end{zgled}

Dve števili za kateri je $D(a,b)=1$ imenujemo tuji števili.

Najmanjši skupni večratnik dveh ali več števil je najmanjše tako število, ki je deljivo z obema (ali vsemi). (v razcepu vzamemo največje potence)

\begin{zgled}
    Poišči najmanjši skupni večkratnik $v(52,130)$ in vse večkratnike obeh števil.
\end{zgled}

Velja $a\cdot b=D(a,b)\cdot v(a,b)$.

\begin{zgled}
    Neca gre v knižnico vsakih 14 dni, Nace vsakih 10 dni. Če se srečata danes, čez koliko dni se bosta srečala. \textcolor{gray}{Iščemo lcm}
\end{zgled}

\begin{zgled}
    Klemen peče piškote. Če jih v vsako vrsto zloži 7 mu dva ostaneta. Če naredi eno vrsto več in jih v vsako položi le 6 ima vse vrste polne. Koliko piškotov ima. \textcolor{gray}{Najprej jih imamo $k\cdot 7 +2$, nato $(k+1)\cdot 6$. $k$ je število vrst}
\end{zgled}

\end{document}