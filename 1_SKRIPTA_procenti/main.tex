\documentclass{article}
\usepackage[utf8]{inputenc}
\usepackage{pgfplots}
\pgfplotsset{width=10cm,compat=1.9}
\usepackage{amsmath,amssymb,amsthm}
\usepackage{graphicx}
\usepackage{float}
\usepackage{blindtext}
\usepackage{hyperref}
\usepackage{verbatim}
\usepackage{gensymb}
\usepackage{enumerate}
\usepackage{xcolor}
\usepackage{graphicx}
\hypersetup{
    colorlinks=true,
    linkcolor=blue,
    filecolor=magenta,      
    urlcolor=cyan,
    pdftitle={Overleaf Example},
    pdfpagemode=FullScreen,
    }
\usepackage[slovene]{babel}

\setlength{\parindent}{0pt}
\setlength{\parskip}{4pt}

\newcommand{\Lim}[1]{\raisebox{0.5ex}{\scalebox{0.8}{$\displaystyle \lim_{#1}\;$}}}
\newcounter{example}[section]
\newenvironment{example}[1][]{\refstepcounter{example}\par\medskip
   \noindent \textbf{Naloga~\theexample. #1} \rmfamily}{\medskip}

\newtheorem*{zgled}{Zgled}

\title{Racionalna števila}
\author{Bor Bregant}
\date{\vspace{-5ex}}

\begin{document}

\maketitle

\section*{Procentni račun}

\begin{zgled}
    Osnovna živilska košarica stane 100 evrov. Koliko bo stala naslednje leto, če se bo podražila za $7\%$?
    \begin{align*}
        \text{cena}_{\text{nova}} &= \text{cena}_{\text{stara}}+\text{podražitev}\\
            &= \text{cena}_{\text{stara}}+7\% \text{od stare cene}\\
            &= \text{cena}_{\text{stara}}+\frac{7}{100}\text{cena}_{\text{stara}}\\
            &= \text{cena}_{\text{stara}}\left(1+\frac{15}{100}\right)\\
    \end{align*}
    Koliko bo stala čez dve leti, če se bo skupno pocenila za $3\%$?
\end{zgled}

\begin{zgled}
    Knjiga se je pocenila za $30\%$, nato pa podražila za $10\%$. Zdaj stane 62 evrov.\\
    Koliko je stala pred spremembami cene?\\
    Koliko odstotna je bila skupna podražitev?\\
    Ali bi bilo vseeno, če se najprej podraži in nato poceni?
\end{zgled}

\end{document}