\documentclass{article}
\usepackage[utf8]{inputenc}
\usepackage{pgfplots}
\pgfplotsset{width=10cm,compat=1.9}
\usepackage{amsmath,amssymb,amsthm}
\usepackage{gensymb}
\usepackage{graphicx}
\usepackage{float}
\usepackage{xcolor}
\usepackage{blindtext}
\usepackage{hyperref}
\usepackage[margin=1in]{geometry}
\hypersetup{
    colorlinks=true,
    linkcolor=blue,
    filecolor=magenta,      
    urlcolor=cyan,
    pdftitle={Overleaf Example},
    pdfpagemode=FullScreen,
    }
\usepackage[slovene]{babel}

\newcounter{example}[section]
\newenvironment{example}[1][]{\refstepcounter{example}\par\medskip
   \noindent \textbf{Naloga~\theexample. #1} \rmfamily}{\medskip}

\newtheorem*{zgled}{Zgled}

\title{Kombinatorika}
\author{Bor Bregant}
\date{\vspace{-5ex}}

\begin{document}

\maketitle
\begin{zgled}
    Na koliko načinov lahko za ravno mizo sedi sedem povabljencev?
\end{zgled}

\textbf{Osnovni izrek kombinatorike ali pravilo produkta}: Če neki proces lahko razdelimo na $k$ zaporednih faz in je prva od faz izvedljiva na $n_1$ načinov, druga na $n_2$ načinov, tretja na $n_3$ načinov, … in $k$-ta na $n_k$ načinov (kjer so izbori med sabo neodvisni), je celotni proces izvedljiv na $n=n_1\cdot \ldots n_k$ načinov.

\begin{zgled}
    Na koliko načinov se lahko oblečemo, če imamo na razpolago dva para čevljev, pet srajc, troje hlač in štiri kravate?\\
    Na koliko načinov lahko damo nase pokrivalo, če imamo na voljo 3 klobuke in dve čepici (\textit{nezdružljivost}).
\end{zgled}


\textbf{Pravilo vsote}: Če izbiramo med $n_1$ možnostmi iz prve množice izborov ali $n_2$ možnostmi iz druge množice naborov in tako naprej (kjer so izbori med sabo neodvisni in nezdružljivi) do $k$-tega nabora, potem je vseh izborov $M=n_1 + \ldots n_k$.

\begin{zgled}
    Do ŠKG lahko pridemo z avtobusi številk 3 ali 5, kjer v obeh primerih naprej prestopimo na 1, 8 ali 25, lahko pa gremo s kolesom ali z avtom. Na koliko različnih način lahko pridemo do šole?
\end{zgled}

\begin{zgled}
    Koliko je vseh različnih metov, če petkrat zapored vržemo kovanec. Predstavi s \textbf{kombinatoričnim drevesom}.
\end{zgled}

\begin{example}
    DN 232a, 248, 254
\end{example}

\section{Permutacije}

Razporeditve $n$ različnih elementov na $n$ mest, kjer je vrstni red pomemben imenujemo permutacije $n$ elementov. Teh možnosti je $P_n = n! = n (n-1)(n-2)\cdot \ldots \cdot 2\cdot 1$.

\begin{zgled}
    Izračunaj $4!$ in $\frac{n!}{(n-1)!}$.
\end{zgled}

\begin{zgled}
    Koliko besed lahko sestavimo iz črk $ABCDE$, če:\\
    Ni omejitev?\\
    Besede se morajo začeti na $D$\\
    Besede se ne začnejo niti na $A$ niti na $E$\\
    Besede se ne končajo na $DA$
\end{zgled}

\begin{zgled}
    Šestčlanska družina gre v kino. Na koliko načinov se lahko usede v vrsto, če sedita starša skupaj in otroci skupaj. Kaj pa če starša sedita na obeh koncih, otroci pa med njima.
\end{zgled}

\begin{zgled}
    Sedem otrok stoji v vrsti. Na koliko načinov jih lahko prestavimo, če trije najbolj živahni ne smejo biti vsi skupaj.
\end{zgled}

\begin{zgled}
    Preštejmo vse permutacije črk besede $ANANAS$.
\end{zgled}

Permutacij $n$ elementov, kjer se en ponavlja $k_1$-krat, drugi $k_2$-krat in tako naprej, je $P_n ^{k_1,\ldots k_r}=\frac{n!}{k_1 ! \cdots k_r !}$

\begin{zgled}
    Koliko besed iz črk $BOMBAZ$ se ne začne s črko $A$.
\end{zgled}

\begin{example}
    DN 262, 263, 273, 267, 281, 292.
\end{example}

\section{Variacije}

$n$ elementov razporejamo na $r$ mest ($r<n$).

\begin{zgled}
    Na koliko načinov lahko razporedimo 10 dijakov za mizo za 4 osebe.
\end{zgled}

Variacije brez ponavljanja:
\[V_n ^r = N(n-1)(n-2)\cdots (n-r+1)=\frac{n!}{(n-r)!}\]

\begin{zgled}
    Koliko trimestnih števil lahko sestavimo s števkami $1,2,5,8$, če se števke ne smejo ponavljati. Kaj pa će dodamo $0$.
\end{zgled}

\begin{zgled}
    Ponavljanje
\end{zgled}

Variacije s ponavljanjem



\end{document}