\documentclass{article}
\usepackage[utf8]{inputenc}
\usepackage{pgfplots}
\pgfplotsset{width=10cm,compat=1.9}
\usepackage{amsmath,amssymb,amsthm}
\usepackage{gensymb}
\usepackage{graphicx}
\usepackage{float}
\usepackage{xcolor}
\usepackage{blindtext}
\usepackage{hyperref}
\usepackage[margin=1in]{geometry}
\hypersetup{
    colorlinks=true,
    linkcolor=blue,
    filecolor=magenta,      
    urlcolor=cyan,
    pdftitle={Overleaf Example},
    pdfpagemode=FullScreen,
    }
\usepackage[slovene]{babel}

\newcounter{example}[section]
\newenvironment{example}[1][]{\refstepcounter{example}\par\medskip
   \noindent \textbf{Naloga~\theexample. #1} \rmfamily}{\medskip}

\newtheorem*{zgled}{Zgled}

\title{Obrestni račun}
\author{Bor Bregant}
\date{\vspace{-5ex}}

\begin{document}

\maketitle

\section{Terminologija}

\begin{itemize}
    \item Glavnica: denarna vrednost, ki jo damo banki v hrambo ali dolg v primeru izposoje. $G$
    \item Obrestna mera (v \%): Vpliva na povečanje oz. zmanjšanje glavnice. $p$
    \item Čas obrestovanja (v dneh, mesecih, letih) $n$
    \item Kapitalizacijska oz. obrestovalna doba: Časovno obdobje, po katerem se obresti pripišejo glavnici.
\end{itemize}

Navadno obrestovanje: Obresti ne obrestuje naprej - vezano le na glavnico.
Aritmetično zaporedje - diferenca $\frac{Gp}{100}$
\[G_1 = G+G\cdot\frac{p}{100}\]
\[G_2 = G+2G\frac{p}{100}\]
\[G_n=G+nG\frac{p}{100}\]

Obrestno obrestovanje: Obresti obrestujejo - glavnice tvorijo geometrijsko zaporedje:\\
\[G_1=g+G\frac{p}{100}=G(1+\frac{p}{100})=Gr; r=1+\frac{p}{100}\ \text{obrestovalni faktor}\]
\[G_2=G(q+\frac{p}{100})^2=Gr^2\]
\[G_n=G(1+\frac{p}{100})^n=Gr^n\]

\begin{zgled}
    Na banko položimo $1000E$ pri $5\%$ obrestni meri za $40$ let. Za koliko se v tem času spremeni glavnica pri navadnem in obrestnem obrestovanju?
\end{zgled}
\begin{zgled}
    Ali se bolj splača: Takoj dobiti $100E$ in čez dve leti še $50E$, ali pa takoj dobiti $50E$ in čez eno leto še $110E$, če je pripis obresti konec leta po obrestni meri $5\%$.
\end{zgled}

Načelo ekvivalence glavnic:

\begin{example}
    \textcolor{red}{331ab}, \fbox{339a}bcd, \fbox{343}-347, \fbox{355a}b, 361ac, \fbox{372a}bc,
\end{example}



\end{document}